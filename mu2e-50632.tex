% -*- mode:flyspell; mode:latex -*-
\documentclass[12pt]{article}
\addtolength{\oddsidemargin} {-0.885in}
\addtolength{\textwidth}{1.75in}
\addtolength{\evensidemargin}{-0.8in}


\usepackage[latin1]{inputenc}
\usepackage[T1]{fontenc}
\usepackage[english]{babel}
\usepackage{graphicx}
\usepackage{float}
%% \usepackage{siunitx}

%% \usepackage{gensymb}


\usepackage{tikz}
\usepackage{[caption}
\usetikzlibrary{arrows}
\usetikzlibrary{decorations.markings}
\usetikzlibrary{decorations.pathmorphing}
% \usepackage[absolute,overlay]{textpos}
% \usepackage{onimage}

\usepackage{tabularx}
\usepackage{times}
\usepackage{graphics}

% \usepackage{subfigure}
% \usepackage{scalefnt}
%
% \renewcommand\thesubfigure{\arabic{subfigure}}

\usepackage{amsmath}
\usepackage{hyperref}
\usepackage{hhline}
\usepackage{subfig}
\usepackage{color}
\usepackage[all]{hypcap}

\usepackage[normalem]{ulem}  % for striking out
% \usepackage{fancyhdr}
% \pagestyle{fancy}
% \fancyhead[C]{}
% \fancyhead[L] {\it{Mu2e-doc-29670-v1.0} }
%%%%%%%%%%%%%%%%%%%%%%%%%%%%%%%%%%%%%%%%%%%%%%%%%%%%%%%%%%%%%%%%%%%%%%%%%%%%%%
% use natbib - biblatex not available on Mu2e interactive nodes
%%%%%%%%%%%%%%%%%%%%%%%%%%%%%%%%%%%%%%%%%%%%%%%%%%%%%%%%%%%%%%%%%%%%%%%%%%%%%%
\usepackage[square,sort,comma,numbers]{natbib}

% location of the .bib files: env var BIBINPUTS (~/library/bibliography)

% \usepackage[backend=biber, style=numeric-comp, sorting=ynt] {biblatex}
% \addbibresource{clfv.bib}

% \addbibresource{stntuple.bib}
% \addbibresource{mu2e_web.bib}
% \addbibresource{radiative_pion_capture.bib}

\graphicspath{{figures/}}
%%%%%%%%%%%%%%%%%%%%%%%%%%%%%%%%%%%%%%%%%%%%%%%%%%%%%%%%%%%%%%%%%%%%%%%%%%%%%%
% if portability is needed, include the ~/tex/commands/commands.tex file locally
% for example, copy it locally, rename into local_commands.tex, and include
%%%%%%%%%%%%%%%%%%%%%%%%%%%%%%%%%%%%%%%%%%%%%%%%%%%%%%%%%%%%%%%%%%%%%%%%%%%%%%
% \include{local_commands}
\include{commands}
%%%%%%%%%%%%%%%%%%%%%%%%%%%%%%%%%%%%%%%%%%%%%%%%%%%%%%%%%%%%%%%%%%%%%%%%%%%%%%
\begin{document}

\begin{titlepage}
  \begin{flushright}
    \bf {MU2E/PHYSICS/50632} \\
    version 1.01
    \today
 \end{flushright}

  \vspace{1cm}

  \begin{center}
    {\Large \bf On the choice of the TS antiproton absorption elements \\

      \vspace{0.3in}
      (summary of the considerations)
    }

    \vspace{1cm}
    P.Murat(FNAL)

    \footnote{\texttt{Fermilab; e-mail: murat@fnal.gov}}
    \vspace{0.3cm}

    \vspace{0.8cm}
  \end{center}

  \begin{abstract}
    \vspace{0.2in}
    This note presents a summary of consideration underlying the choice
    parameters of the pbar absorbers in the Mu2e beamline.
  \end{abstract}

\end{titlepage}
% \frontmatter
% \chapter*{Abstract}
%
% \addcontentsline{toc}{chapter}{Abstract}
%
% \mainmatter
%
{\tableofcontents}

%%%%%%%%%%%%%%%%%%%%%%%%%%%%%%%%%%%%%%%%%%%%%%%%%%%%%%%%%%%%%%%%%%%%%%%%%%%%%%%
%\chapter{Calibration}
%%%%%%%%%%%%%%%%%%%%%%%%%%%%%%%%%%%%%%%%%%%%%%%%%%%%%%%%%%%%%%%%%%%%%%%%%%%%%%%
% \input{input_data}

%%%%%%%%%%%%%%%%%%%%%%%%%%%%%%%%%%%%%%%%%%%%%%%%%%%%%%%%%%%%%%%%%%%%%%%%%%%%%%%

\newpage
\section {Revision History and TODO items}

\begin{itemize}
\item
  v1.01: inital version
\end{itemize}

%%%%%%%%%%%%%%%%%%%%%%%%%%%%%%%%%%%%%%%%%%%%%%%%%%%%%%%%%%%%%%%%%%%%%%%%%%%%%%
\newpage
\section {Introduction}

Antiproton annihilation in the stopping target (ST) is one of the background
sources to the search for $\mu \to e$ conversion. The background is suppressed
by several absorption elements:

\begin{itemize}
\item
  an absoption window in front of the TS1 collimator.
  Its main purpose of the TS1 window is to reduce the flux of antiprotons
  reaching the TS3, and through that, the contribution of delayed RPC.
\item
  a ``collar'' on the bottom of the TS1 collimator. The collar suppresses
  the high-momentum, p > 100 \MeVc\  component of the antiproton background
\item
  an absorption window and an additional wedge-shaped absorber in the
  middle of TS3, in between the TS3u and TS3d collimators stop the
  antiprotons with momenta below $\simeq 100$ \MeVc, and by virtue of doing
  that serve as a source of the delayed RPC background.
\end{itemize}

The window in the middle of TS3 serves an additional purpose of separating
the PS and DS vacuum regions which have different vacuum requirements,
$< 10^{-5}$ and $< 10^{-4}$ torr respectively.

In case of an accidental vacuum loss, the window should be able withstand
the 50 psi pressure differential \cite{XXXX}.

To withstand a differential of 1 atm, a D=45 cm Ti window should
be about 300 um thick.

%
\section{TS1 window}


%%% Local Variables:
%%% mode: latex
%%% TeX-master: t
%%% End:

%
\section{TS1 collar}


%%% Local Variables:
%%% mode: latex
%%% TeX-master: t
%%% End:

%
\section{TS3 window and wedge}


%%% Local Variables:
%%% mode: latex
%%% TeX-master: t
%%% End:


%%%%%%%%%%%%%%%%%%%%%%%%%%%%%%%%%%%%%%%%%%%%%%%%%%%%%%%%%%%%%%%%%%%%%%%%%%%%%% 
\section {Summary}

We presented a summary of considerations underlying the choice of parameters 
of the Mu2e antiproton absorbers. 

%%%%%%%%%%%%%%%%%%%%%%%%%%%%%%%%%%%%%%%%%%%%%%%%%%%%%%%%%%%%%%%%%%%%%%%%%%%%%%
%
%%%%%%%%%%%%%%%%%%%%%%%%%%%%%%%%%%%%%%%%%%%%%%%%%%%%%%%%%%%%%%%%%%%%%%%%%%%%%%
\newpage
\bibliographystyle{unsrtnat}
\bibliography{clfv,mu2e_internal_notes,mu2e_pbar_notes}

% \include{appendix_a}
% \include{appendix_b}

\end{document}
